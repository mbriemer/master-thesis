% !TeX program = pdflatex
% !TeX TXS-program:compile = txs:///pdflatex/
% !TeX TS-program = pdflatex
% !BIB program = biber
% !TeX TXS-program:bibliography = txs:///biber




%%%%%%%%%%%%%%%%%%%%%%%%%%%%%%%%%%%%%%%%%%%%%%%%%
%%  GRID-BASED TYPESETTING AS FAR AS POSSIBLE  %%
%%%%%%%%%%%%%%%%%%%%%%%%%%%%%%%%%%%%%%%%%%%%%%%%%


\flushbottom

\newlength{\origbaselineskip}
\setlength{\origbaselineskip}{\baselineskip}

\ifdef{\linesperpagedesired}
	{}% If already defined, do nothing.
	{\newcommand{\linesperpagedesired}{42}}
	% Number of text lines per page, see https://en.wikipedia.org/wiki/Gutenberg_Bible

\newcommand{\linesperpagecurrent}{\numexpr (\textheight - \topskip) / \baselineskip + 1 \relax}  % Integer division!
\makeatletter
%\newcommand{\newbaselinestretch}{\strip@pt\dimexpr (\linesperpagecurrent pt - 1pt) / (\linesperpagedesired - 1)}
\newcommand{\newbaselinestretch}{1.1}
\makeatother
\linespread{\newbaselinestretch}
\newlength{\newbaselineskip}
\setlength{\newbaselineskip}{
	\dimexpr (\textheight - \topskip) / (\linesperpagedesired - 1)
}
\setlength{\textheight}{\dimexpr \numexpr \linesperpagedesired - 1 \relax \newbaselineskip + \topskip}  % to prevent too small \textheigtht due to rounding errors
\newlength{\newparindent}
\setlength{\newparindent}{1.15\newbaselineskip}%
\AtBeginDocument{%
	\setlength{\baselineskip}{\newbaselineskip}%
	\setlength{\parindent}{\newparindent}
	\setlength{\lineskiplimit}{0pt}
		% Prevents increased line spacing in response to spacious inline formulas.
}
\newlength{\abovedisplayauxskip}
\newlength{\belowdisplayauxskip}
\setlength{\abovedisplayauxskip}{0pt plus 0.5\baselineskip}
\setlength{\belowdisplayauxskip}{0pt plus 0.5\baselineskip}
\newcommand{\predisplaycmd}{%
	\ifvmode\else\unskip\fi%
	\nopagebreak[2]%
	\vspace{\abovedisplayauxskip}%
}
\makeatletter
\def\@itemize@name{itemize}
\def\@enumerate@name{enumerate}
\def\@description@name{description}
\newcommand{\postdisplaycmd}{%
	\ifx\@currenvir\@itemize@name%
	\else%
		\ifx\@currenvir\@enumerate@name%
		\else%
			\ifx\@currenvir\@description@name%
			\else%
				\vskip\belowdisplayauxskip%
			\fi%
		\fi%
	\fi%
	\noindent%
}
\makeatother
\AtBeginEnvironment {align}      {\predisplaycmd}
\AfterEndEnvironment{align}      {\postdisplaycmd}
\AtBeginEnvironment {align*}     {\predisplaycmd}
\AfterEndEnvironment{align*}     {\postdisplaycmd}
\AtBeginEnvironment {alignat}    {\predisplaycmd}
\AfterEndEnvironment{alignat}    {\postdisplaycmd}
\AtBeginEnvironment {alignat*}   {\predisplaycmd}
\AfterEndEnvironment{alignat*}   {\postdisplaycmd}
\AtBeginEnvironment {displaymath}{\predisplaycmd}
\AfterEndEnvironment{displaymath}{\postdisplaycmd}
\AtBeginEnvironment {eqnarray}   {\predisplaycmd}
\AfterEndEnvironment{eqnarray}   {\postdisplaycmd}
\AtBeginEnvironment {eqnarray*}  {\predisplaycmd}
\AfterEndEnvironment{eqnarray*}  {\postdisplaycmd}
\AtBeginEnvironment {equation}   {\predisplaycmd}
\AfterEndEnvironment{equation}   {\postdisplaycmd}
\AtBeginEnvironment {equation*}  {\predisplaycmd}
\AfterEndEnvironment{equation*}  {\postdisplaycmd}
\AtBeginEnvironment {flalign}    {\predisplaycmd}
\AfterEndEnvironment{flalign}    {\postdisplaycmd}
\AtBeginEnvironment {flalign*}   {\predisplaycmd}
\AfterEndEnvironment{flalign*}   {\postdisplaycmd}
\AtBeginEnvironment {gather}     {\predisplaycmd}
\AfterEndEnvironment{gather}     {\postdisplaycmd}
\AtBeginEnvironment {gather*}    {\predisplaycmd}
\AfterEndEnvironment{gather*}    {\postdisplaycmd}
\AtBeginEnvironment {multiline}  {\predisplaycmd}
\AfterEndEnvironment{multiline}  {\postdisplaycmd}
\AtBeginEnvironment {multiline*} {\predisplaycmd}
\AfterEndEnvironment{multiline*} {\postdisplaycmd}
% \setlength{\topskip}{\newbaselineskip pt}

\setlength{\parskip}{0pt}  % Prevents whitespace from being added between paragraphs
%\setlength{\parskip}{0pt plus 0.0001pt}
% Add whitepsace between paragraphs only in emergency cases




%%%%%%%%%%%%%%%%%%%%%%%%%%%%%%%%%
%%  FORMATTING OF MATHEMATICS  %%
%%%%%%%%%%%%%%%%%%%%%%%%%%%%%%%%%


%\setlength{\mathindent}{\newparindent}  % Only for option ``fleqn''

% Prevent stretching of spaces around operators in inline formulas; see ==>
%  http://tex.stackexchange.com/questions/83746/keeping-the-distance-between-mathematical-symbols-consistent
% \thinmuskip=3.0mu (already without glue)
\medmuskip=1\medmuskip	% Formerly 4.0mu plus 2.0mu minus 4.0mu -> 4.0mu
\thickmuskip=1\thickmuskip	% Formerly 5.0mu plus 5.0mu -> 5.0mu
% <==

% Increase spacing around relational symbols (<, =, >, \le, \ge, \equiv; +, -, \times, etc.) in display formulae ==>
\newcommand{\regularmu}{\thickmuskip= 5mu \medmuskip=4mu}
\newcommand{\thickmu}  {\thickmuskip=10mu \medmuskip=5mu}
\AtBeginEnvironment{align}      {\thickmu}
\AtBeginEnvironment{align*}     {\thickmu}
\AtBeginEnvironment{alignat}    {\thickmu}
\AtBeginEnvironment{alignat*}   {\thickmu}
\AtBeginEnvironment{displaymath}{\thickmu}
\AtBeginEnvironment{eqnarray}   {\thickmu}
\AtBeginEnvironment{eqnarray*}  {\thickmu}
\AtBeginEnvironment{equation}   {\thickmu}
\AtBeginEnvironment{equation*}  {\thickmu}
\AtBeginEnvironment{flalign}    {\thickmu}
\AtBeginEnvironment{flalign*}   {\thickmu}
\AtBeginEnvironment{gather}     {\thickmu}
\AtBeginEnvironment{gather*}    {\thickmu}
\AtBeginEnvironment{multiline}  {\thickmu}
\AtBeginEnvironment{multiline*} {\thickmu}
% <==

% Use the bm (= boldmath) package for better support of setting math in bold ==>
% Prevent the "Too many math fonts used" error:
\newcommand{\bmmax}{0}
\newcommand{\hmmax}{0}
\usepackage{bm}
% <==

% Allow paragraph breaks after a display equation
\makeatletter
\predisplaypenalty=\@medpenalty
\postdisplaypenalty=0
\makeatother




%%%%%%%%%%%%%%%%%%%%%%%%%%%%%
%%  LAYOUT AND SECTIONING  %%
%%%%%%%%%%%%%%%%%%%%%%%%%%%%%

% Disable single lines that start a~paragraph at the end of a~page (widows/Schusterjungen)
% and disable single lines at the end of a~paragraph that start a~new page (orphans/Hurenkinder):
\usepackage[all]{nowidow}

\AtBeginDocument{%
	% Reduce the amount of white space after a period (and enforce this throughout the entire document, because babel's \selectlanguage{...} resets \frenchspacing):
	\let\nonfrenchspacing=\frenchspacing
	\frenchspacing
	\sloppy%  % Prevent overfull hboxes (at the expense of more uneven whitespace)
}

\ifxetex
	\usepackage[protrusion=true, expansion=false]{microtype}
\else
	\usepackage[protrusion=true, expansion=false, kerning=true]{microtype}
\fi

% Reduce by how much an interword space can shrink; see
% http://tex.stackexchange.com/questions/19236/how-to-change-the-interword-spacing,
% http://tex.stackexchange.com/questions/88991/what-do-different-fontdimennum-mean, and
% http://mirrors.ctan.org/macros/latex/contrib/everysel/everysel.pdf:
\AddToHook{selectfont}{%
	\spaceskip = \fontdimen2\font plus \fontdimen3\font minus 0.75\fontdimen4\font%
}

% Let even relatively big floats (long tables, spacious figures) be placed on text pages ==>
\renewcommand{\textfraction}{0.05}
\renewcommand{\topfraction}{0.95}
\renewcommand{\bottomfraction}{0.95}
% <== See http://tex.stackexchange.com/questions/39017/how-to-influence-the-position-of-float-environments-like-figure-and-table-in-lat/39020#39020

% Set margins of block quotes
% ==>
\renewenvironment{quote}%
  {\list{}{\leftmargin=\parindent \rightmargin=\parindent}%
   \linespread{\newbaselinestretch}\item[]\itshape\small}%
  {\endlist}
\renewenvironment{quotation}%
  {\list{}{\leftmargin=\parindent \rightmargin=\parindent
           \listparindent=\parindent \parsep=0pt}%
   \item[]}%
  {\endlist}
% <==

\usepackage{csquotes}

\usepackage{nameref}
\makeatletter
\newcommand*{\currentname}{\@currentlabelname}
\makeatother




%%%%%%%%%%%%%%%%%%%%%%%%%%%%%%%
%%  FORMATTING OF FOOTNOTES  %%
%%%%%%%%%%%%%%%%%%%%%%%%%%%%%%%


\usepackage[multiple, bottom, norule, splitrule, marginal]{footmisc}
% \renewcommand{\multfootsep}{,\,}
% AER-like style: no regular footnote rule, half-page splitrule
% ==>
\setlength{\footnotemargin}{0.85\newparindent}%
\renewcommand{\footnotelayout}{\hspace{0.15\newparindent}}
\renewcommand{\hangfootparindent}{\newparindent}
\preto{\footnote}{\setlength{\parindent}{\newparindent}}
% Add some space around the footnoterule:
\let\oldfootnoterule\footnoterule
\addtolength{\skip\footins}{\bigskipamount}
\AtBeginDocument{%
	\renewcommand{\splitfootnoterule}{{\hrule width 0.5\textwidth}}%
	\renewcommand{\footnoterule}{\oldfootnoterule\medskip}%
}
% <==
%\setlength{\footnotesep}{0.8\baselineskip}

% Chicago Manual of Style (16th edition):
% ``Note reference numbers in text are set as superior (superscript) numbers.
% In the notes themselves, they are normally full size, not raised, and followed by a period.''
% (also REStud and JEEA style)
% ==>
\usepackage{xstring}
\newlength{\textparindent}
\setlength{\textparindent}{\parindent}
\newlength{\templength}
\makeatletter
\let \@makefntextorig \@makefntext
    % Saving the original definition so we can reuse it if necessary.
\newcommand{\@makefntextcustom}[1]{%
	\parindent 2\textparindent%
	\hspace{-\textparindent}%
	\settowidth{\templength}{0}%
	\ifnum\value{footnote}<10 \hspace{\templength}\else\fi%
	\thefootnote.\enskip #1%
}
\renewcommand{\@makefntext}[1]{\@makefntextcustom{#1}}
\makeatother
%\usepackage{scrextend}
%\newlength{\footnoteflmargin}
%\setlength{\footnoteflmargin}{\parindent}
%% \deffootnote[\footnoteflmargin]{0pt}{0pt}{\thefootnotemark.\:\,}
%\deffootnote[\footnoteflmargin]{0pt}{\footnoteflmargin}{}
%\renewcommand{\footnote}[1]{%
%	\footnoteorig{%
%		\ifnum\value{footnote}<10 \phantom{0}\else\fi%
%		\thefootnotemark.\enskip%
%		#1%
%	}%
%}
% <==
\let \thefootnoteorig \thefootnote
\DefineFNsymbols*{star}{%
	{$\mathrm{\star}$}{$\mathrm{\star\star}$}{$\mathrm{\ddagger}$}%
	{$\mathrm{\ddagger\ddagger}$}{\S}{\S\S}{\P}{\P\P}{$\mathrm{\|}$}{$\mathrm{\|\|}$}%
}
\setfnsymbol{star}




%%%%%%%%%%%%%%%%%%%%%%%%%%
%%  FIGURES AND TABLES  %%
%%%%%%%%%%%%%%%%%%%%%%%%%%


\usepackage[singlelinecheck=on]{caption}
\DeclareCaptionLabelSeparator{periodlargespace}{.\:\:}
\captionsetup{
	singlelinecheck=on,
	figureposition=below,
	tableposition=above,
	format = plain,
	labelsep = periodlargespace,
	margin = 0pt,
	font = {sf, small},
	labelfont = {sf, bf, small},
	justification = justified
}

% Referencing ``subfigures'' (i.e., individual panels of which a figure is comprised):
%\usepackage{subfigure}
%\usepackage{subfig}
% Justus says this is the newer and preferable package.
% However, see https://tex.stackexchange.com/questions/13625/subcaption-vs-subfig-best-package-for-referencing-a-subfigure:
\usepackage{subcaption}

% Packages for creating better-looking tables
\usepackage{booktabs}
\setlength{\cmidrulewidth}{.035em}
\setlength{\lightrulewidth}{.035em}
\setlength{\heavyrulewidth}{.09em}
\setlength{\abovetopsep}{-5pt}
\addtolength{\aboverulesep}{1.5pt}	% Make tables a little more spacious
\addtolength{\belowrulesep}{1.5pt}
% \setlength{\belowbottomsep}{-2pt}

\usepackage{tabularx}	% Provides environment tabularx to adjust width of tables
% \usepackage{tabulary}	% For some reason, tabulary doesn't obey the width argument ...
% Emulate the "tabulary" column types:
\newcolumntype{C}{>{\centering\arraybackslash}X}
\newcolumntype{J}{>{\arraybackslash}X}
\newcolumntype{L}{>{\RaggedRight\arraybackslash}X}
\newcolumntype{R}{>{\RaggedLeft\arraybackslash}X}
% \usepackage[flushleft]{threeparttable}	% Provides the tablenotes environment

% Remove superfluous whitespace at beginning and end of table rows -->
\LetLtxMacro{\oldtabular}{\tabular}
\LetLtxMacro{\endoldtabular}{\endtabular}
\RenewDocumentEnvironment{tabular}{O{c} m}{%
	\oldtabular[#1]{@{}#2@{}}%
}{%
	\endoldtabular%
}
\LetLtxMacro{\oldtabularx}{\tabularx}
\LetLtxMacro{\endoldtabularx}{\endtabularx}
\RenewDocumentEnvironment{tabularx}{m O{c} m}{%
	\oldtabularx{#1}[#2]{@{}#3@{}}%
}{%
	\endoldtabularx%
}

% Package for long tables that span multiple pages
\usepackage{xltabular}

\usepackage{siunitx}[=v2]
	% Allows, among others, for alignment of decimal numbers in tables at the decimal point.
\sisetup{
	detect-all,
	round-integer-to-decimal = true,
	group-digits             = true,
	group-minimum-digits     = 5,
	group-separator          = {\kern 1pt},
	table-align-text-pre     = false,
	table-align-text-post    = false,
	input-signs              = + -,
	input-symbols            = {*} {**} {***} \sigstar,
	input-open-uncertainty 	 = ,
	input-close-uncertainty  = ,
	retain-explicit-plus
}
\newcolumntype{T}[1]{@{}S[table-format = #1, table-space-text-pre = {***}, table-space-text-post = {***}]}
% Fix incompatibility of siunitx (v2018-05-17) with FiraSans (v2019-06-06),
% based on https://tex.stackexchange.com/questions/213605/siunitx-does-not-detect-semi-bold-font
% ==>
\ExplSyntaxOn\makeatletter
\newcommand{\thisseries}{\f@series}
\cs_set_protected:Npn \__siunitx_detect_font_weight_text: {%
	\let\origmdseries\mdseries@sf%
	\let\origbfseries\bfseries@sf%
	\let\currentseries\f@series%
	\edef\XcurrentseriesX{/\f@series/}%
		% Store the current fontseries but enclose it in some kind of delimiter,
		% because otherwise one-letter fontseries may erroneously triger \boldmath
		% (for instance, ``m'' is contained in ``semibold'')
	% Use \boldmath for any weight above semibold:
	\tl_if_in:noTF
		{ /sb/ /b/ /bx/ /eb/ /ub/ /bold/ /extrabold/ /ultrabold/ /heavy/ /black/ /demibold/ /semibold/ }
		{ \XcurrentseriesX }
		{% if included in the above list: switch to \mathversion{bold}
			\cs_set:Nn \__siunitx_font_weight: {%
				\boldmath%
				\let\bfseries@sf\currentseries%
					% necessary because siunitx in some way accesses
					% \bfseries@sf from the mweights package
				\fontseries{\bfseries@sf}\selectfont%
			}%
			\let\bfseries@sf\origbfseries%  % restore \bfseries@sf
		}
		{% if not: use \mathversion{normal}
			\cs_set:Nn \__siunitx_font_weight: {%
				\unboldmath
				\let\mdseries@sf\currentseries%
					% necessary because siunitx in some way accesses
					% \bfseries@sf from the mweights package
				\fontseries{\mdseries@sf}\selectfont%
			}%
			\let\mdseries@sf\origmdseries%  % restore \mdseries@sf
		}%
}
\makeatother\ExplSyntaxOff
% <==

% Ability to add footnotes to tables:
% ==>
\usepackage[restart, breakwithin, indentafter]{parnotes}
	% BEWARE: For some reason, \parnotes removes the vertical space before the following section/the indent of the following paragraph if not included in the table itself!
\renewcommand{\parnotevskip}{0pt}
% \renewcommand{\parnoteintercmd}{\\}
\renewcommand{\theparnotemark}{\alph{parnotemark}}
\renewcommand{\parnotefmt}[1]{\footnotesize\noindent\justify #1\par}
%\renewcommand{\parnotefmt}[1]{\footnotesize\rmfamily%
%	\noindent\rule{\linewidth}{1pt}\\%
%	\noindent#1\par%
%	\noindent\rule{\linewidth}{1pt}\\%
%}
%\renewcommand{\parnoteintercmd}{\;$\bullet$\;}
% <==

\usepackage{makecell}

\usepackage{longtable}

\usepackage{multirow}

% Redefine figure environment so that all figures are centered
% ==>
\makeatletter
\let\oldfigure\figure
\def\figure{\@ifnextchar[\figure@i \figure@ii}
\def\figure@i[#1]{\oldfigure[#1]\centering}
\def\figure@ii{\oldfigure\centering}
\makeatother
% <==

\usepackage[font={sf, small}]{floatrow}	% Set the font in tables to sansserif small
\floatsetup[table]{style=Plaintop}
\renewcommand{\floatfootskip}{\smallskipamount}
\newcommand{\tablenotes}[2][Notes:]{%
	\floatfoot*{%
		\setlength{\baselineskip}{11pt}%
		\textit{#1} #2%
	}%
}
\newcommand{\figurenotes}[2][Notes:]{%
	\floatfoot{%
		\setlength{\baselineskip}{11pt}%
		\vspace{-\floatfootskip}%
		\vspace{\medskipamount}%
		\\[-\baselineskip]
		\textit{#1} #2%
	}%
}
% \usepackage{float}	% Allows the inclusion of figures inside a minipage	% Do not use in combination with floatrow.
\usepackage{placeins} % improve placing of floats (figures, tables), provides \FloatBarrier

% Adjust the minimum distance between a float (figure, table) and the body text:
\setlength{\textfloatsep}{1.5\newbaselineskip plus 0.5\newbaselineskip minus 0.0pt}
% originally, \textfloatsep: 20.0pt plus 2.0pt minus 4.0pt

\usepackage{verbatim}

\usepackage{longtable}

%\usepackage{setspace}

%\DeclareCaptionLabelFormat{REStudTable}{\MakeUppercase{\tablename}~#2}
%\DeclareCaptionTextFormat{REStudTableT}{\textit{#1}}
%\captionsetup[table]{labelformat=REStudTable, textformat=REStudTableT}




%%%%%%%%%%%%%%%%%%%%%%%%%%%
%%  FORMATTING OF LISTS  %%
%%%%%%%%%%%%%%%%%%%%%%%%%%%


\usepackage[inline]{enumitem}
% General settings:
\setlist{leftmargin=\parindent, listparindent=\parindent, itemsep=\smallskipamount, parsep=0pt}
%\setlist[1]{topsep=\medskipamount, partopsep=0pt}
% Type-specific settings
%\setlist[enumerate]{labelwidth=\parindent, labelindent=0pt, labelsep=!, align=left}
\setlist[enumerate]{leftmargin=\parindent, labelsep=*}
	% Fine as long as the list does not include more than 9 items.
\setlist[enumerate, 1]{label=(\arabic*), labelindent=-0.5pt}
\setlist[enumerate, 2]{label=\alph*., align=right}
	% Taken from the Chicago Manual of Style (16th ed., Section 6.126)
\setlist[enumerate, 3]{label=\roman*., align=right, widest*=3, labelsep=0.3\parindent}
\setlist[itemize]{labelsep=0.435\parindent}
\setlist[description]{font=\rmfamily\normalsize}




%%%%%%%%%%%%%%%%%%%%%%%%%%%%%%
%%  FORMATTING OF THEOREMS  %%
%%%%%%%%%%%%%%%%%%%%%%%%%%%%%%


\newtheoremstyle{Standard}% name
	{\topsep}    % Space above: Use \topsep to make the space identical to the one around lists
	{\topsep}    % Space below
	{\itshape}   % Body font
	{}           % Indent amount (empty = no indent, \parindent = paragraph indent)
	{\bfseries}  % Theorem head font
	{.}          % Punctuation after theorem head
	{.5em}       % Space after theorem head: " " = normal interword space; \newline = linebreak
	{\thmname{#1}\thmnumber{\:#2}\thmnote{\bfseries\upshape\ (#3)}}
		% Theorem head spec (changed such that also the ``theorem note'' is printed in boldface)

\theoremstyle{Standard}
\newtheorem{theorem}{Theorem}
\newtheorem{corollary}[theorem]{Corollary}
\newtheorem{lemma}[theorem]{Lemma}
\newtheorem{proposition}[theorem]{Proposition}
\newtheorem{hypothesis}{Hypothesis}
\newtheorem{result}{Result}

\theoremstyle{definition}
\newtheorem{definition}[theorem]{Definition}
\newtheorem{example}[theorem]{Example}
\newtheorem{conjecture}[theorem]{Conjecture}

% Make numbering chapter-specific if we are compiling the dissertation template:
\makeatletter
\@ifclassloaded{book}{%
	\numberwithin{theorem}{chapter}%
	%\numberwithin{corollary}{chapter}%
	%\numberwithin{lemma}{chapter}%
	%\numberwithin{proposition}{chapter}%
	\numberwithin{hypothesis}{chapter}%
	\numberwithin{result}{chapter}%
	%\numberwithin{definition}{chapter}%
	%\numberwithin{example}{chapter}%
	%\numberwithin{conjecture}{chapter}%
}
\makeatother



%%%%%%%%%%%%%%%%%%%%%%%%%%%%%%%%%%%%%%%%%%%%%%%%%%%%%%%%%
%%  AUTOMATIC CAPITALIZATION OF HEADINGS AND CAPTIONS  %%
%%%%%%%%%%%%%%%%%%%%%%%%%%%%%%%%%%%%%%%%%%%%%%%%%%%%%%%%%


\usepackage{mfirstuc-english}
% \gMFUnocap{xxx} adds ``xxx'' to the words not to be capitalized
% Do not capitalize prepositions:
\gMFUnocap{about}
\gMFUnocap{at}
\gMFUnocap{against}
\gMFUnocap{around}
\gMFUnocap{between}
\gMFUnocap{by}
\gMFUnocap{from}
\gMFUnocap{on}
\gMFUnocap{over}
\gMFUnocap{per}
\gMFUnocap{to}
\gMFUnocap{versus}
\gMFUnocap{vs.}
\gMFUnocap{vis-\`a-vis}
\gMFUnocap{within}
\gMFUnocap{without}

\ifcase 0
	% 0 for disabling auto-capitalization, 1 for enabling auto-capitalization of headings, 2 for headings + figure/table captions
	% Case 0: Make capitalization commands ineffective
	\renewcommand{\capitalisewords}[1]{#1}
	\renewcommand{\ecapitalisewords}[1]{#1}
	\renewcommand{\xcapitalisewords}[1]{#1}
\or
	% Case 1:
	% Add auto-capitalization to table of contents -->
	\let\SavedContentsline\contentsline
	\renewcommand{\contentsline}[4]{%
		\SavedContentsline{#1}{\capitalisewords{#2}}{#3}{#4}%
	}
	% <--
\else
	% Case 2: Also auto-capitalize figure and table captions:
	% Add auto-capitalization to table of contents -->
	\let\SavedContentsline\contentsline
	\renewcommand{\contentsline}[4]{%
		\SavedContentsline{#1}{\capitalisewords{#2}}{#3}{#4}%
	}
	% <--
	\LetLtxMacro{\SavedCaption}{\caption}
	\RenewDocumentCommand{\caption}{ O{\shortcaption} m }{%
		\def\shortcaption{%
			\xcapitalisewords{%
				% \TestForPunct{%
				#2%
				% }%
			}%
		}%
		\SavedCaption[#1]{%
			\xcapitalisewords{%
				% \TestForPunct{%
				#2%
				% }%
			}%
		}%
	}
\fi




%%%%%%%%%%%%%%%%%%%%%%
%%  APPENDIX STYLE  %%
%%%%%%%%%%%%%%%%%%%%%%


\usepackage[title, titletoc]{appendix}  % allows, e.g., for appendices within chapters
% Patch a bug in appendix.sty,
% see https://github.com/wspr/herries-press/issues/34#issuecomment-935770824:
\makeatletter
\patchcmd\@resets@pp{\xdef}{\def}{}{\fail}
\patchcmd\@resets@ppsub{\xdef}{\def}{}{\fail}
\makeatother
\usepackage{chngcntr}  % to innclude section numbers/letters in the figure/table/equation counters

\AtBeginEnvironment{appendices}{%
	\counterwithin{figure}{section}%
	\counterwithin{table}{section}%
	\counterwithin{equation}{section}%
}
\AtBeginEnvironment{subappendices}{%
	\counterwithin{figure}{section}%
	\counterwithin{table}{section}%
	\counterwithin{equation}{section}%
}

% Revoke the changes at the end of the (sub)appendices environment
% if we are compiling the dissertation template:
\makeatletter
\@ifclassloaded{book}{%
	\AfterEndEnvironment{appendices}{%
		\counterwithout{figure}{section}%
		\counterwithin{figure}{chapter}%
		\counterwithout{table}{section}%
		\counterwithin{table}{chapter}%
		\counterwithout{equation}{section}%
		\counterwithin{equation}{chapter}%
	}%
	\AfterEndEnvironment{subappendices}{%
		\counterwithout{figure}{section}%
		\counterwithin{figure}{chapter}%
		\counterwithout{table}{section}%
		\counterwithin{table}{chapter}%
		\counterwithout{equation}{section}%
		\counterwithin{equation}{chapter}%
	}%
}{}
\makeatother